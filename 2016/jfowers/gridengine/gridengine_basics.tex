\documentclass[]{article}
\usepackage{lmodern}
\usepackage{amssymb,amsmath}
\usepackage{ifxetex,ifluatex}
\usepackage{fixltx2e} % provides \textsubscript
\ifnum 0\ifxetex 1\fi\ifluatex 1\fi=0 % if pdftex
  \usepackage[T1]{fontenc}
  \usepackage[utf8]{inputenc}
\else % if luatex or xelatex
  \ifxetex
    \usepackage{mathspec}
    \usepackage{xltxtra,xunicode}
  \else
    \usepackage{fontspec}
  \fi
  \defaultfontfeatures{Mapping=tex-text,Scale=MatchLowercase}
  \newcommand{\euro}{€}
\fi
% use upquote if available, for straight quotes in verbatim environments
\IfFileExists{upquote.sty}{\usepackage{upquote}}{}
% use microtype if available
\IfFileExists{microtype.sty}{%
\usepackage{microtype}
\UseMicrotypeSet[protrusion]{basicmath} % disable protrusion for tt fonts
}{}
\usepackage[margin=1in]{geometry}
\usepackage{color}
\usepackage{fancyvrb}
\newcommand{\VerbBar}{|}
\newcommand{\VERB}{\Verb[commandchars=\\\{\}]}
\DefineVerbatimEnvironment{Highlighting}{Verbatim}{commandchars=\\\{\}}
% Add ',fontsize=\small' for more characters per line
\usepackage{framed}
\definecolor{shadecolor}{RGB}{248,248,248}
\newenvironment{Shaded}{\begin{snugshade}}{\end{snugshade}}
\newcommand{\KeywordTok}[1]{\textcolor[rgb]{0.13,0.29,0.53}{\textbf{{#1}}}}
\newcommand{\DataTypeTok}[1]{\textcolor[rgb]{0.13,0.29,0.53}{{#1}}}
\newcommand{\DecValTok}[1]{\textcolor[rgb]{0.00,0.00,0.81}{{#1}}}
\newcommand{\BaseNTok}[1]{\textcolor[rgb]{0.00,0.00,0.81}{{#1}}}
\newcommand{\FloatTok}[1]{\textcolor[rgb]{0.00,0.00,0.81}{{#1}}}
\newcommand{\CharTok}[1]{\textcolor[rgb]{0.31,0.60,0.02}{{#1}}}
\newcommand{\StringTok}[1]{\textcolor[rgb]{0.31,0.60,0.02}{{#1}}}
\newcommand{\CommentTok}[1]{\textcolor[rgb]{0.56,0.35,0.01}{\textit{{#1}}}}
\newcommand{\OtherTok}[1]{\textcolor[rgb]{0.56,0.35,0.01}{{#1}}}
\newcommand{\AlertTok}[1]{\textcolor[rgb]{0.94,0.16,0.16}{{#1}}}
\newcommand{\FunctionTok}[1]{\textcolor[rgb]{0.00,0.00,0.00}{{#1}}}
\newcommand{\RegionMarkerTok}[1]{{#1}}
\newcommand{\ErrorTok}[1]{\textbf{{#1}}}
\newcommand{\NormalTok}[1]{{#1}}
\usepackage{graphicx}
\makeatletter
\def\maxwidth{\ifdim\Gin@nat@width>\linewidth\linewidth\else\Gin@nat@width\fi}
\def\maxheight{\ifdim\Gin@nat@height>\textheight\textheight\else\Gin@nat@height\fi}
\makeatother
% Scale images if necessary, so that they will not overflow the page
% margins by default, and it is still possible to overwrite the defaults
% using explicit options in \includegraphics[width, height, ...]{}
\setkeys{Gin}{width=\maxwidth,height=\maxheight,keepaspectratio}
\ifxetex
  \usepackage[setpagesize=false, % page size defined by xetex
              unicode=false, % unicode breaks when used with xetex
              xetex]{hyperref}
\else
  \usepackage[unicode=true]{hyperref}
\fi
\hypersetup{breaklinks=true,
            bookmarks=true,
            pdfauthor={James Owers},
            pdftitle={Gridengine Basics},
            colorlinks=true,
            citecolor=blue,
            urlcolor=blue,
            linkcolor=magenta,
            pdfborder={0 0 0}}
\urlstyle{same}  % don't use monospace font for urls
\setlength{\parindent}{0pt}
\setlength{\parskip}{6pt plus 2pt minus 1pt}
\setlength{\emergencystretch}{3em}  % prevent overfull lines
\setcounter{secnumdepth}{5}

%%% Use protect on footnotes to avoid problems with footnotes in titles
\let\rmarkdownfootnote\footnote%
\def\footnote{\protect\rmarkdownfootnote}

%%% Change title format to be more compact
\usepackage{titling}

% Create subtitle command for use in maketitle
\newcommand{\subtitle}[1]{
  \posttitle{
    \begin{center}\large#1\end{center}
    }
}

\setlength{\droptitle}{-2em}
  \title{Gridengine Basics}
  \pretitle{\vspace{\droptitle}\centering\huge}
  \posttitle{\par}
  \author{James Owers}
  \preauthor{\centering\large\emph}
  \postauthor{\par}
  \predate{\centering\large\emph}
  \postdate{\par}
  \date{27 June 2016}



\begin{document}

\maketitle


{
\hypersetup{linkcolor=black}
\setcounter{tocdepth}{2}
\tableofcontents
}
\textbf{TL;DR}: To use \texttt{James} or \texttt{Charles} servers as if
you were \texttt{ssh}ing into them as before, just \texttt{ssh renown}
then \texttt{qlogin}.

\section{Who is this for}\label{who-is-this-for}

People who use the \texttt{James} or \texttt{Charles} servers. Until now
we have \texttt{ssh}'d into the servers but now \texttt{ssh} access has
been removed from all but a few. Now in place is Son of a Grid Engine
(\texttt{SGE}) to control access to servers. This guide shows you how to
continue much like before and how to use basic \texttt{SGE} commands.

Son of a Grid Engine is an open source version of Univa Grid Engine (née
Oracle Grid Engine (née Sun Grid Engine))

\subsection{Useful references}\label{useful-references}

\begin{itemize}
\itemsep1pt\parskip0pt\parsep0pt
\item
  \href{https://arc.liv.ac.uk/trac/SGE}{SGE project site}
\item
  \href{http://arc.liv.ac.uk/SGE/htmlman/manuals.html}{SGE
  documentation}
\item
  \texttt{man qsub} from within \texttt{renown}
\item
  \href{http://star.mit.edu/cluster/docs/0.92rc2/guides/sge.html}{MIT
  SGE introduction}
\end{itemize}

\section{Getting started}\label{getting-started}

Log in to the gridengine machine \texttt{renown}

\begin{Shaded}
\begin{Highlighting}[]
\CommentTok{## If not on a dice machine}
\KeywordTok{kinit} \NormalTok{s0816700}
\KeywordTok{aklog}
\KeywordTok{ssh} \NormalTok{-K s0816700@staff.ssh.inf.ed.ac.uk}
\CommentTok{# ssh -K s0816700@student.ssh.inf.ed.ac.uk}
\KeywordTok{ssh} \NormalTok{renown}
\end{Highlighting}
\end{Shaded}

You will be in your home directory, in my case, \texttt{/home/s0816700}.
We can see that lots of space has been added to the \texttt{/home/}
directory (\texttt{/mnt/cdt\_gridengine\_home})

\begin{Shaded}
\begin{Highlighting}[]
\KeywordTok{df} \NormalTok{-h}
\end{Highlighting}
\end{Shaded}

\begin{verbatim}
Filesystem                                     Size  Used Avail Use% Mounted on
/dev/vda1                                       24G  5.1G   18G  23% /
devtmpfs                                       2.0G     0  2.0G   0% /dev
tmpfs                                          2.0G     0  2.0G   0% /dev/shm
tmpfs                                          2.0G  9.4M  2.0G   1% /run
tmpfs                                          2.0G     0  2.0G   0% /sys/fs/cgroup
/etc/glusterfs/gv0.vol                         147G   84G   57G  60% /disk/glusterfs/gv0
charles11.inf.ed.ac.uk:/cdt-gridengine-common  385G  264M  365G   1% /mnt/cdt_gridengine_common
anne.inf.ed.ac.uk:/cdt-gridengine-home         2.7T  432G  2.2T  17% /mnt/cdt_gridengine_home
/dev/vda4                                      7.6G   65M  7.1G   1% /var/cache/afs
AFS                                            2.0T     0  2.0T   0% /afs
tmpfs                                          396M     0  396M   0% /run/user/656624
tmpfs                                          396M     0  396M   0% /run/user/28328
tmpfs                                          396M     0  396M   0% /run/user/1559549
tmpfs                                          396M     0  396M   0% /run/user/1421660
\end{verbatim}

\emph{Not covered here:} how to run parallel jobs and writing to the
distributed file system `Gluster'. For information on running parallel
jobs using SGE, see the latter half of the
\href{http://star.mit.edu/cluster/docs/0.92rc2/guides/sge.html}{MIT SGE
introduction}.

\section{Basic SGE commands}\label{basic-sge-commands}

\subsection{Interactive session on a node (just like
\texttt{ssh}ing)}\label{interactive-session-on-a-node-just-like-sshing}

\begin{Shaded}
\begin{Highlighting}[]
\KeywordTok{qlogin}
\end{Highlighting}
\end{Shaded}

Useful options:

\begin{itemize}
\item
  specify a specific node

\begin{Shaded}
\begin{Highlighting}[]
\KeywordTok{qlogin} \NormalTok{-l h=charles14}
\end{Highlighting}
\end{Shaded}
\item
  specify resource must have a GPU

\begin{Shaded}
\begin{Highlighting}[]
\KeywordTok{qlogin} \NormalTok{-l gpu=1}
\end{Highlighting}
\end{Shaded}
\end{itemize}

\subsection{Submit a script to the
queue}\label{submit-a-script-to-the-queue}

\begin{Shaded}
\begin{Highlighting}[]
\KeywordTok{qsub} \NormalTok{myscript.sh}
\end{Highlighting}
\end{Shaded}

OUTPUT two files containing the stdout and sterr
{[}script-name{]}.o{[}jobnr{]} and {[}script-name{]}.e{[}jobnr{]}, and
whatever files or directories your script creates

\subsection{View status of your subitted
jobs}\label{view-status-of-your-subitted-jobs}

\begin{Shaded}
\begin{Highlighting}[]
\KeywordTok{qstat}
\end{Highlighting}
\end{Shaded}

OUTPUT

\begin{verbatim}
job-ID  prior   name       user         state submit/start at     queue                          slots ja-task-ID 
-----------------------------------------------------------------------------------------------------------------
     15 0.55500 long_sleep s0816700     r     06/03/2016 22:54:38 all.q@charles11.inf.ed.ac.uk       1 

state = *qw*/**r** for *queued and waiting*/**running**
\end{verbatim}

\subsection{Deleting Jobs}\label{deleting-jobs}

\begin{Shaded}
\begin{Highlighting}[]
\KeywordTok{qdel}
\end{Highlighting}
\end{Shaded}

\subsection{Viewing Node Status}\label{viewing-node-status}

\texttt{qhost}

OUTPUT

\begin{verbatim}
HOSTNAME                ARCH         NCPU NSOC NCOR NTHR  LOAD  MEMTOT  MEMUSE  SWAPTO  SWAPUS
----------------------------------------------------------------------------------------------
global                  -               -    -    -    -     -       -       -       -       -
anne                    lx-amd64       64    4   64   64  0.02  995.6G    8.5G   31.2G     0.0
charles01               lx-amd64       32    2   16   32  1.01   62.7G    8.3G   31.3G     0.0
charles02               lx-amd64       32    2   16   32  0.27   62.7G    3.7G   31.3G     0.0
charles03               lx-amd64       32    2   16   32  0.01   62.7G    3.2G   31.3G     0.0
charles04               lx-amd64       32    2   16   32  0.04   62.7G    2.5G   31.3G     0.0
charles05               lx-amd64       32    2   16   32 13.61   62.7G    6.0G   31.3G     0.0
charles06               -               -    -    -    -     -       -       -       -       -
charles07               -               -    -    -    -     -       -       -       -       -
charles08               -               -    -    -    -     -       -       -       -       -
charles09               -               -    -    -    -     -       -       -       -       -
charles10               -               -    -    -    -     -       -       -       -       -
charles11               lx-amd64       24    2   12   24  0.01   62.8G    2.6G   31.4G     0.0
charles12               lx-amd64       24    2   12   24  0.01   62.8G    2.4G   31.4G     0.0
charles13               lx-amd64       24    2   12   24  0.01   62.8G    2.6G   31.4G     0.0
charles14               lx-amd64       24    2   12   24  0.01   62.8G    2.6G   31.4G     0.0
\end{verbatim}

\section{Example: Running an IPython Notebook and accessing it from
outside
DICE}\label{example-running-an-ipython-notebook-and-accessing-it-from-outside-dice}

\begin{enumerate}
\def\labelenumi{\arabic{enumi}.}
\item
  Setup python virtual environment with IPython Notebook installed

  \begin{itemize}
  \itemsep1pt\parskip0pt\parsep0pt
  \item
    Tip: install it in your home directory on DICE
  \end{itemize}
\item
  \texttt{qlogin} to your server of choice
\item
  Check GPU use with \texttt{nvidia-smi}
\item
  activate your python virtual environment (you'll need to
  \texttt{kinit} \& \texttt{aklog} if this is located on your DICE home
  as recommended)

\begin{Shaded}
\begin{Highlighting}[]
\KeywordTok{source} \NormalTok{/afs/inf.ed.ac.uk/user/s08/s0816700/venv/nolearn/bin/activate}
\end{Highlighting}
\end{Shaded}
\item
  create a password hash using python

\begin{Shaded}
\begin{Highlighting}[]
\CharTok{from} \NormalTok{IPython.lib }\CharTok{import} \NormalTok{passwd}
\NormalTok{passwd()}
\NormalTok{exit}
\end{Highlighting}
\end{Shaded}
\item
  start the IPython Notebook

\begin{Shaded}
\begin{Highlighting}[]
\KeywordTok{longjob} \NormalTok{-28day -c }\StringTok{'ipython notebook  --ip="*" --NotebookApp.password=sha1:0880f873e98f:9ddab235858c92ea9a2e02877b5b324bf091ef93 --no-browser --port=1337'}\KeywordTok{`}
\end{Highlighting}
\end{Shaded}
\item
  access the notebook

  \begin{itemize}
  \itemsep1pt\parskip0pt\parsep0pt
  \item
    From within forum simply browse to \texttt{http://charles13:1337}
    (replacing charles13 with where you were)
  \item
    Outside the forum either:

    \begin{itemize}
    \itemsep1pt\parskip0pt\parsep0pt
    \item
      first kinit \& aklog then, ssh port forward charles13:1337 back to
      your computer:
      \texttt{ssh -K -L 8889:charles13:1337 s0816700@staff.ss.inf.ed.ac.uk}
      then go to \texttt{http://localhost:8889}
    \item
      or connect via VPN and navigate to \texttt{...} TODO: fill this
      in!
    \end{itemize}
  \end{itemize}
\end{enumerate}

{\textbf{WARNING}}: If anyone finds/hacks your password\ldots{}they have
access to your filesystem

\textbf{AWESOME WIN}: this longjob process allows continual access to
your filesystem after the original afs ticket expires

\section{Current issues}\label{current-issues}

\begin{itemize}
\itemsep1pt\parskip0pt\parsep0pt
\item
  Automatic resource allocation doesn't appear to take into account GPU
  use\ldots{}
\item
  \ldots{}working with Charles and Iain Rae on that
\item
  IPython Notebook solution isn't very secure
\item
  If you are running a script containing \texttt{longjob} using
  \texttt{qsub}, how is your kerberos ticket handled
\end{itemize}

\section{Tips}\label{tips}

\begin{itemize}
\itemsep1pt\parskip0pt\parsep0pt
\item
  when logged in to \texttt{renown} type \texttt{q} then double tap
  \texttt{tab} to get a list of the commands for use!
\item
  \texttt{nvidia-smi} lets you check the GPU use on a server - if the
  command doesn't work then the server you are on doesn't have a GPU;
  try to login to another server; you can specify a specific server with
  \texttt{qlogin -l h=charles14}
\end{itemize}

\end{document}
